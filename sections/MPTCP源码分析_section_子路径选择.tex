\section{子路径选择}
\subsection{简述子路径选择}
支持MPTCP的链路中存在多条子路径,因此在发送数据的时候需要选择最优路径来进行操作。MPTCP利用内核通知链对MPTCP中各子路径进行增加路径、删除路径、修改路径优先级的操作。MPTCP根据相应的策略进行路径选择。
\subsection{路径选择的代码实现}
路径选择的关键在于从多个子路径中选择其中一个进行数据的发送。此过程通过下面的函数实现:
\small\begin{verbatim}
"net/mptcp/mptcp_sched.c" line 114 of 496
114 static struct sock *get_available_subflow(struct sock *meta_sk,
115                       struct sk_buff *skb,
116                       bool zero_wnd_test)
117 {
118     struct mptcp_cb *mpcb = tcp_sk(meta_sk)->mpcb;   //MPTCP内核实现的三个组成部分之一:Multi-path control bock(mpcb)
119     struct sock *sk, *bestsk = NULL, *lowpriosk = NULL, *backupsk = NULL;
120     u32 min_time_to_peer = 0xffffffff, lowprio_min_time_to_peer = 0xffffffff;
121     int cnt_backups = 0;
122
123     /* if there is only one subflow, bypass the scheduling function */
124     if (mpcb->cnt_subflows == 1) {
125         bestsk = (struct sock *)mpcb->connection_list;
126         if (!mptcp_is_available(bestsk, skb, zero_wnd_test))
127             bestsk = NULL;
128         return bestsk;
129     }
130
131     /* Answer data_fin on same subflow!!! */
132     if (meta_sk->sk_shutdown & RCV_SHUTDOWN &&
133         skb && mptcp_is_data_fin(skb)) {
134         mptcp_for_each_sk(mpcb, sk) {
135             if (tcp_sk(sk)->mptcp->path_index == mpcb->dfin_path_index &&
136                 mptcp_is_available(sk, skb, zero_wnd_test))
137                 return sk;
138         }
139     }
\end{verbatim}\normalsize
第124行的代码是处理只有一个子路径的特殊情况。第132行是对关闭操作进行了处理,在收包过程中对接收关闭包的子路径进行记录,然后回包的时候使用相同的子路径。
\small\begin{verbatim}
"net/mptcp/mptcp_input.c" line 876 of 2293
875     /* Record it, because we want to send our data_fin on the same path */
876     if (tp->mptcp->map_data_fin) {
877         mpcb->dfin_path_index = tp->mptcp->path_index;
878         mpcb->dfin_combined = !!(sk->sk_shutdown & RCV_SHUTDOWN);
879     }
\end{verbatim}\normalsize
第876行是对子路径关闭的处理,关于data\_fin可以参考https://tools.ietf.org/html/rfc6824\#section-3.3.3。下面的代码遍历mpcb下管理的sk,选择最合适的sk。
\small\begin{verbatim}
"net/mptcp/mptcp_sched.c" line 142 of 496
141     /* First, find the best subflow */
142     mptcp_for_each_sk(mpcb, sk) {
143         struct tcp_sock *tp = tcp_sk(sk);
144
145         if (tp->mptcp->rcv_low_prio || tp->mptcp->low_prio)
146             cnt_backups++;
147
148         if ((tp->mptcp->rcv_low_prio || tp->mptcp->low_prio) &&
149             tp->srtt < lowprio_min_time_to_peer) {
150             if (!mptcp_is_available(sk, skb, zero_wnd_test))
151                 continue;
152
153             if (mptcp_dont_reinject_skb(tp, skb)) {
154                 backupsk = sk;
155                 continue;
156             }
157
158             lowprio_min_time_to_peer = tp->srtt;
159             lowpriosk = sk;
160         } else if (!(tp->mptcp->rcv_low_prio || tp->mptcp->low_prio) &&
161                tp->srtt < min_time_to_peer) {
162             if (!mptcp_is_available(sk, skb, zero_wnd_test))
163                 continue;
164
165             if (mptcp_dont_reinject_skb(tp, skb)) {
166                 backupsk = sk;
167                 continue;
168             }
169
170             min_time_to_peer = tp->srtt;
171             bestsk = sk;
172         }
173     }
\end{verbatim}\normalsize
从上面的代码可以看出,选择sk的依据有5条:
\begin{itemize}
  \item “tp$\rightarrow$mptcp$\rightarrow$rcv\_low\_prio ” which stands for priority level in the remote machine.
  \item "tp$\rightarrow$mptcp$\rightarrow$low\_prio" which stands for priority level in the local machine.
  \item whether the skb has already been enqueued into this subsocket
  \item "tp$\rightarrow$srtt" stands for smoothed round trip time.
  \item mptcp\_is\_available() check the congestion of this subflow.
\end{itemize}
\small\begin{verbatim}
"net/mptcp/mptcp_sched.c" line 175 of 496
175     if (mpcb->cnt_established == cnt_backups && lowpriosk) {
176         sk = lowpriosk;
177     } else if (bestsk) {
178         sk = bestsk;
179     } else if (backupsk) {
180         /* It has been sent on all subflows once - let's give it a
181          * chance again by restarting its pathmask.
182          */
183         if (skb)
184             TCP_SKB_CB(skb)->path_mask = 0;
185         sk = backupsk;
186     }
187
188     return sk;
189 }
\end{verbatim}\normalsize
第176行的情况说明所有的子路径都处于backup状态。而第178行则是存在bestsk。

\subsection{结论}
\begin{itemize}
  \item 控制路径选择的因素有下面四个:
    \begin{enumerate}
        \item 此路径在对端机器的优先级
        \item 此路径在本地的优先级
        \item 此次发送的skb是否已经使用此路径发送过,不能在同一路径重复发送。
        \item tp$\rightarrow$srtt
    \end{enumerate}
  \item 调整 low\_prio 和 rcv\_low\_prio 可以通过下面命令:
    \begin{enumerate}
    \item ip link set dev eth0 multipath backup
    \item 当通讯双方的一端被设置为backup后,可以通过MP\_PRIO
    \item 通知对端。具体内容可以参考:https://tools.ietf.org/html/rfc6824\#section-3.3.8
    \end{enumerate}
  \item srtt的调整通过函数 tcp\_ack\_update\_rtt 实现。
\end{itemize}