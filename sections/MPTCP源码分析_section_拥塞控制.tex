\section{拥塞控制}
\subsection{简述拥塞控制}
MPTCP的拥塞控制对TCP的拥塞控制的线性增加阶段进行了修改,而慢启动,快速重传、快速恢复都没有改变。每条子路径拥有自己的cwnd,MPTCP的拥塞算法主要关心cwnd的改变。
\subsection{拥塞算法设计原则}
\begin{itemize}
  \item MPTCP的Throughput 要达到MPTCP中所有子路径中最好的一条路径
  \item MPTCP应该和普通TCP一样从共享资源中获得相同资源
  \item MPTCP中的流量将从拥塞的子路径转移到不拥塞的路径。
\end{itemize}
\subsection{算法理解}
MPTCP的各个子路径运行着正常的TCP,因此直观的我们可以在每条子路径上运行自己的拥塞控制算法,但是这样就违背了设计原则2,这样的效果是MPTCP的吞吐量就会超过其他正常TCP。因此有如下EWTCP算法(n表示路径数量,$a=\frac{1}{\sqrt{n}}$):
\begin{description}
  \item[对于路径r上的每一个ACK] 窗口增加的数值$W_r=\frac{a}{W_r}$
  \item[对于路径r上的每一个LOSS] 窗口减少的数值$W_r=\frac{W_r}{2}$
\end{description}
a的取值参考~\cite{Honda2009Multipath},这样,MPTCP就把每次cwnd的增加分摊到各个不同的子路径上,这样MPTCP就和正常TCP有着相同的吞吐量。但是这样的算法设计存在问题,不能有效的利用网络环境,我们应该根据设计原则3,将流量移动到拥塞情况最少的路径上去。因此有如下COUPLED算法($W_t$是所有子流的窗口之和):
\begin{description}
  \item[对于路径r上的每一个ACK] 窗口增加的数值$W_r=\frac{1}{W_t}$
  \item[对于路径r上的每一个LOSS] 窗口减少的数值$W_r=\frac{W_t}{2}$
\end{description}
此算法让各个子路径的拥塞窗口的变化联系起来,比如有两条路径,一条路径上面拥塞导致导致丢包严重,那么不断的减少$W_t/2$,这样的话,就将流量从拥塞的路径移动到不拥塞的路径上。但是,这个算法存在两个问题:
\begin{itemize}
  \item 如果拥塞的子路径完全没有流量,我们就无从得知这条子路径上拥塞情况以后是不是会改善。
  \item 没有考虑到RTT的的因素,比如对于一个智能手机来说,3G网络和WIFI相比丢包率更低,而RTT更大。
\end{itemize}
但是因为3G的拥塞情况更好,因此流量大部分会通过3G网络。而3G网络的吞吐量可能小于WIFI的吞吐量。因此提出MPTCP的拥塞控制算法($W_r$的路径r的当前窗口值):
\begin{description}
  \item[对于路径r上的每一个ACK] 窗口增加的数值$W_r=\min(\frac{a}{W_t},\frac{1}{W_r})$
  \item[对于路径r上的每一个LOSS] 窗口减少的数值$W_r=\frac{W_r}{2}$
\end{description}
此算法通过 min操作来遵守设计原则2,通过a来保证各个子路径上都有适当的流量,从而达到设计原则1和3,详细的算法描述可以参考~\cite{Wischik2011Design}。
\subsection{MPTCP的内核实现}
MPTCP会在接收每一个ACK的时候,计算算法中的a。调用情况如下:
\small\begin{verbatim}
     tcp_ack()
               =>tcp_ca_event()
                    =>cwnd_event()
                         =>mptcp_ccc_cwnd_event()
\end{verbatim}\normalsize

在tcp\_ack函数中也会增加cwnd,调用情况如下:
\small\begin{verbatim}
     tcp_ack()
               =>tcp_cong_avoid()
                    =>cong_avoid()
                         => mptcp_ccc_cong_avoid()
\end{verbatim}\normalsize
